% Mengubah keterangan `Abstract` ke bahasa indonesia.
% Hapus bagian ini untuk mengembalikan ke format awal.
\renewcommand\abstractname{Abstrak}

\begin{abstract}

  % Ubah paragraf berikut sesuai dengan abstrak dari penelitian.
 Penelitian ini menghadirkan pendekatan baru untuk meningkatkan mobilitas bagi pasien ALS dengan mengembangkan sistem kontrol pergerakan kursi roda menggunakan CNN \emph{(Convolutional Neural Network)}, yang berfokus pada pengenalan pose mata, dan diimplementasikan pada platform Intel NUC. Studi ini didorong oleh tantangan yang dihadapi oleh pasien ALS, yang meskipun kehilangan mobilitas fisik, tetap dapat menggerakkan mata. Sistem yang diusulkan bertujuan untuk memberikan kemandirian yang lebih besar dan peningkatan kualitas hidup bagi individu tersebut. Metodologi penelitian ini terdiri dari beberapa komponen kunci: pengambilan dan pengolahan gambar mata, ekstraksi fitur yang relevan, estimasi dan klasifikasi posisi mata, serta eksekusi sistem kontrol pada \textit{Next Unit of Computing (NUC)}. Dengan memanfaatkan MediaPipe untuk pengenalan pose secara \emph{realtime} dan kemampuan komputasi Intel NUC, studi ini menjanjikan peningkatan yang signifikan dalam otonomi pengguna kursi roda dengan gangguan mobilitas.  Model klasifikasi menunjukkan kinerja yang sangat baik berdasarkan hasil evaluasi confusion matrix, dengan akurasi, \emph{precision, recall, dan f-1 score} sebesar 100\%. Performa model pada jarak 30 dan 50 cm, model memiliki akurasi tertinggi yaitu 100\%. Model memiliki kinerja yang cukup baik di berbagai tingkat pencahayaan, dengan akurasi tertinggi 100\% pada pencahayaan 131 Lux. Sistem mempertahankan FPS yang stabil pada laptop dan Intel NUC, menunjukkan kinerja yang lebih baik pada laptop. Waktu respons motor rata-rata untuk perintah "Kanan," "Kiri," dan "Mundur" di bawah 0,25 detik, sedangkan untuk perintah "Maju" dan "Stop" sekitar 0,43 detik. Motor kursi roda menunjukkan waktu output yang konsisten di setiap kelas perintah, dengan standar deviasi terendah pada perintah "Maju" (0,117), menunjukkan keandalan sistem dalam memberikan respons yang stabil dan konsisten. Penelitian ini tidak hanya berkontribusi pada bidang teknologi bantu, tetapi juga berpotensi menjadi model untuk inovasi masa depan dalam solusi mobilitas bagi individu dengan berbagai disabilitas fisik.

\end{abstract}

% Mengubah keterangan `Index terms` ke bahasa indonesia.
% Hapus bagian ini untuk mengembalikan ke format awal.
\renewcommand\IEEEkeywordsname{Kata kunci}

\begin{IEEEkeywords}

% Ubah kata-kata berikut dengan kata kunci dari tugas akhir
Kata Kunci: \emph{Teknologi Bantu, Pengenalan Pose, Kontrol Kursi Roda, CNN, MediaPipe}

\end{IEEEkeywords}