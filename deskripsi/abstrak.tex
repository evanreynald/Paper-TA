% Mengubah keterangan `Abstract` ke bahasa indonesia.
% Hapus bagian ini untuk mengembalikan ke format awal.
\renewcommand\abstractname{Abstrak}

\begin{abstract}

  % Ubah paragraf berikut sesuai dengan abstrak dari penelitian.
  Penelitian ini menghadirkan pendekatan baru untuk meningkatkan mobilitas bagi pasien
  ALS dengan mengembangkan sistem kontrol pergerakan kursi roda menggunakan MediaPipe,
  yang berfokus pada pengenalan gerakan mata, dan diimplementasikan pada platform Jetson
  Nano. Studi ini didorong oleh tantangan yang dihadapi oleh pasien ALS, yang meskipun
  kehilangan mobilitas fisik, tetap dapat menggerakkan mata. Sistem yang diusulkan bertujuan
  untuk memberikan kemandirian yang lebih besar dan peningkatan kualitas hidup bagi individu
  tersebut. Metodologi penelitian ini terdiri dari beberapa komponen kunci: pengambilan dan
  pengolahan gambar mata, ekstraksi fitur yang relevan, estimasi dan klasifikasi posisi mata, serta
  eksekusi sistem kontrol pada Next Unit of Computing (NUC). Integrasi teknologi visi komputer
  dengan sistem tertanam merupakan batu penjuru penelitian ini, memastikan mekanisme kontrol
  kursi roda yang responsif dan efisien. Dengan memanfaatkan kekuatan MediaPipe untuk
  pengenalan gestur waktu nyata dan kemampuan komputasi Jetson Nano, studi ini menjanjikan
  peningkatan yang signifikan dalam otonomi pengguna kursi roda dengan gangguan mobilitas.
  Penelitian ini tidak hanya berkontribusi pada bidang teknologi bantu, tetapi juga berpotensi
  menjadi model untuk inovasi masa depan dalam solusi mobilitas bagi individu dengan berbagai
  disabilitas fisik.

\end{abstract}

% Mengubah keterangan `Index terms` ke bahasa indonesia.
% Hapus bagian ini untuk mengembalikan ke format awal.
\renewcommand\IEEEkeywordsname{Kata kunci}

\begin{IEEEkeywords}

  % Ubah kata-kata berikut sesuai dengan kata kunci dari penelitian.
  Teknologi Bantu, Pengenalan Gerakan, Kontrol Kursi Roda

\end{IEEEkeywords}
