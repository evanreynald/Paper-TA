% Mengubah keterangan `Abstract` ke bahasa indonesia.
% Hapus bagian ini untuk mengembalikan ke format awal.
% \renewcommand\abstractname{Abstrak}

\begin{abstract}

  % Ubah paragraf berikut sesuai dengan abstrak dari penelitian.
  This thesis presents a novel approach to enhancing mobility for ALS patients by developing
  a wheelchair movement control system using MediaPipe, centered on eye gesture recognition,
  and implemented on a Jetson Nano platform. The study is motivated by the challenges faced
  by ALS patients who, despite losing physical mobility, retain the ability to move their eyes. The
  proposed system aims to provide these individuals with increased independence and improved
  quality of life. The methodology comprises several key components: capturing and processing
  eye images, extracting relevant features, estimating and classifying eye positions, and executing
  the control system on a Next Unit of Computing (NUC). The integration of computer vision
  technology with an embedded system is a cornerstone of this research, ensuring a responsive
  and efficient wheelchair control mechanism. By leveraging the power of MediaPipe for
  real-time gesture recognition and the computational capabilities of Jetson Nano, the study
  promises to significantly enhance the autonomy of wheelchair users with mobility impairments.
  This research not only contributes to the field of assistive technology but also serves as a
  potential model for future innovations in mobility solutions for individuals with various physical
  disabilities.


\end{abstract}

% Mengubah keterangan `Index terms` ke bahasa indonesia.
% Hapus bagian ini untuk mengembalikan ke format awal.
% \renewcommand\IEEEkeywordsname{Kata kunci}

\begin{IEEEkeywords}

  % Ubah kata-kata berikut sesuai dengan kata kunci dari penelitian.
  Assistive Technology, Gesture Recognition, Wheelchair Control

\end{IEEEkeywords}
