% Ubah judul dan label berikut sesuai dengan yang diinginkan.
\section{Kesimpulan}
\label{sec:kesimpulan}

% Ubah paragraf-paragraf pada bagian ini sesuai dengan yang diinginkan.

Berdasarkan hasil pengujian yang telah dilakukan dan dianalisa pada bab sebelumnya, didapatkan beberapa kesimpulan sebagai berikut:

\begin{enumerate}

  \item Penelitian ini berhasil mengembangkan sistem kontrol kursi roda berbasis gestur mata menggunakan MediaPipe dan Intel NUC. Sistem ini mampu mengenali gerakan mata dengan tingkat akurasi tinggi dan menerjemahkannya menjadi perintah untuk mengontrol kursi roda. Sistem ini menawarkan solusi inovatif bagi pasien ALS yang hanya dapat menggerakkan mata.

  \item Model klasifikasi yang digunakan menunjukkan kinerja yang sangat baik berdasarkan hasil evaluasi confusion matrix. Model dapat mengenali berbagai gerakan mata dengan cepat dan konsisten, memastikan kontrol kursi roda yang andal.

  \item Model tetap memiliki performa yang stabil pada jarak 30 hingga 90 cm. Pada jarak 50 cm, model memiliki akurasi tertinggi yaitu 100\%.
  
  \item Model memiliki kinerja yang cukup baik di berbagai tingkat pencahayaan, dengan akurasi tertinggi 100\% pada pencahayaan 131 Lux.
  
  \item Pengujian performa FPS menunjukkan bahwa sistem dapat mempertahankan FPS yang stabil pada laptop (10,415 - 13,972 FPS) dan Intel NUC (8,405 - 12,448 FPS). Variasi ini memastikan bahwa sistem tetap responsif terhadap gerakan mata pengguna.
  
  \item Waktu rata-rata motor untuk perintah "Kanan" adalah 0,4337 detik, menunjukkan stabilitas dan konsistensi dalam merespons perintah. Untuk perintah lainnya seperti "Kiri," "Maju," "Mundur," dan "Stop," waktu respons rata-rata di bawah 0,25 detik, menunjukkan sistem yang responsif.
  
  \item Motor kursi roda memiliki waktu output yang konsisten di setiap kelas perintah, dengan variabilitas yang relatif sempit. Hal ini menunjukkan bahwa sistem kontrol kursi roda mampu memberikan waktu respons yang stabil dan konsisten.

\end{enumerate}