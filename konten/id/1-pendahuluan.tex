% Ubah judul dan label berikut sesuai dengan yang diinginkan.
\section{Pendahuluan}
\label{sec:pendahuluan}

% Ubah paragraf-paragraf pada bagian ini sesuai dengan yang diinginkan.
Disabilitas adalah kondisi dimana tubuh atau pikiran terganggu yang membuat orang dengan kondisi tersebut lebih sulit melakukan aktivitas tertentu dan lebih sulit untuk berinteraksi dengan dunia di sekelilingnya \cite{CDC_2020}. Menurut \textit{World Health Organization}, disabilitas adalah bagian dari diri manusia dan merupakan bagian yang tidak terpisahkan dari pengalaman manusia. Hal ini merupakan hasil dari interaksi antara kondisi kesehatan seperti demensia, kebutaan atau cedera tulang belakang, dan berbagai faktor lingkungan dan pribadi. Diperkirakan 1,3 miliar orang - atau 16\% dari populasi global - mengalami disabilitas yang signifikan saat ini. Jumlah ini terus bertambah karena meningkatnya penyakit tidak menular dan orang yang hidup lebih lama \cite{WHO_2023}.

Salah satu kondisi yang dapat menyebabkan disabilitas yaitu tetraplegia merupakan salah satu kondisi cedera tulang belakang dimana manusia hanya dapat menggerakkan bagian tubuh bagian atas, seperti kepala, leher, dan bahu. Salah satu penyakit yang termasuk dalam kondisi tetraplegia adalah \textit{Amyotrophic Lateral Sclerosis} atau yang biasa disingkat ALS. \textit{Amyotrophic lateral sclerosis} atau ALS, yang sebelumnya dikenal sebagai penyakit Lou Gehrig, adalah kelainan neurologis yang memengaruhi neuron motorik, yaitu sel-sel saraf di otak dan sumsum tulang belakang yang mengontrol gerakan otot dan pernapasan. Ketika neuron motorik mengalami degenerasi dan mati, neuron motorik berhenti mengirimkan pesan ke otot, yang menyebabkan otot melemah, mulai berkedut (fasikulasi), dan mengecil (atrofi). Pada akhirnya, pada penderita ALS, otak kehilangan kemampuannya untuk memulai dan mengontrol gerakan otot yang dibutuhkan untuk berjalan, berbicara, mengunyah dan fungsi lainnya, serta bernapas. ALS bersifat progresif, yang berarti gejalanya memburuk dari waktu ke waktu \cite{NINDS}. 

Dalam stadium terakhir penyakit ALS, dimana rata-rata waktu dalam dua hingga lima tahun \cite{ALS_2023}, pasien ALS biasanya kehilangan kemampuan gerakan fisik termasuk berbicara dan menulis tangan, tetapi untungnya mereka masih bisa menggerakkan mata mereka \cite{Eyesay_2023}. Oleh sebab itu, pasien ALS perlu pendamping dalam melakukan aktivitas sehari-hari, namun ada saatnya dimana mereka harus bisa beraktivitas secara mandiri. Dalam hal tersebut, mereka membutuhkan kursi roda untuk bergerak dalam melakukan aktivitas sehari-hari. Di sisi lain, kursi roda elektrik yang ada pada saat ini, bergantung pada lengan atas pengguna untuk kontrol yang menyulitkan pasien tetraplegia untuk mengendalikannya \cite{9935646}. 

Untuk mengatasi permasalahan tersebut, penting untuk mencari solusi untuk dapat mempermudah sistem kontrol kursi roda sehingga dapat meningkatkan kemandirian pasien ALS. Pendekatan yang menjanjikan adalah pemanfaatan teknologi visi komputer dan sistem tertanam. Visi komputer memungkinkan komputer untuk "melihat" dan memproses citra \cite{TIAN20201}, teknologi ini menggunakan kamera untuk mengidentifikasi, melacak, hingga mengukur target untuk pemrosesan citra lebih lanjut. Visi komputer memberikan kemampuan untuk mengenali dan memahami lingkungan sekitar. Penting untuk menggunakan perangkat keras yang sesuai untuk tujuan tersebut. Baru-baru ini, model pengenalan gambar telah ditanam pada platform target yang sebenarnya, yang merupakan komputer dengan tujuan khusus, seperti perangkat IoT seluler atau kendaraan otonom, dan bukan pada server atau PC yang digunakan di laboratorium. Sistem tertanam umumnya digunakan di lapangan, namun ada persyaratan operasional untuk lingkungan, ukuran fisik, dll \cite{8939843}. Dengan menggabungkan kedua teknologi tersebut, maka pada penelitian ini memberikan solusi yaitu dengan mengembangkan sistem kontrol pergerakan kursi roda yang dapat dikendalikan melalui gerakan mata.

Dalam mewujudkan solusi tersebut, maka penelitian ini difokuskan untuk mengembangkan sistem kontrol kursi roda yang dapat berinteraksi dengan teknologi visi komputer dalam sistem tertanam. NUC \textit{(Next Unit of Computing)} adalah pilihan yang tepat untuk pengembangan sistem tertanam karena efektivitas biaya, kemudahan penggunaan, dan fleksibilitas. Penelitian ini diharapkan dapat menciptakan sistem kontrol kursi roda yang efisien dan responsif dalam mengatasi masalah mobilitas pada penderita ALS.