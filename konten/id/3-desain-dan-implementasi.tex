% Ubah judul dan label berikut sesuai dengan yang diinginkan.
\section{Desain dan Implementasi}
\label{sec:desaindanimplementasi}

\subsection{Cetak Biru Roket}
\label{subsec:cetakbiruroket}

Pada cetak biru yang tertera pada Gambar \ref{fig:cetakbiru}. \lipsum[8]

% Contoh input gambar pada kolom.
\lipsum[9-10]

\subsection{Lorem Ipsum}
\label{subsec:loremipsum}

\lipsum[11]

% Contoh pembuatan tabel.
\begin{table}
  \caption{Contoh tabel sederhana}
  \label{tab:tabelsederhana}
  \centering
  \begin{tabular}{lll}
    \toprule
    Heading1 & Heading2 & Heading3 \\
    \midrule
    One      & Two      & Three    \\
    Four     & Five     & Six      \\
    \bottomrule
  \end{tabular}
\end{table}

% Contoh pembuatan potongan kode.
\begin{lstlisting}[
  language=C++,
  caption={Program halo dunia.},
  label={lst:halodunia}
]
#include <iostream>

int main() {
    std::cout << "Halo Dunia!";
    return 0;
}
\end{lstlisting}

\lipsum[12]

% Contoh pembuatan daftar.
\begin{enumerate}
  \item \lipsum[13][1-4]
  \item \lipsum[13][5-8]
  \item \lipsum[13][9-12]
\end{enumerate}

\lipsum[14-15]