% Ubah judul dan label berikut sesuai dengan yang diinginkan.
\section{Introduction}
\label{sec:introduction}

% Ubah paragraf-paragraf pada bagian ini sesuai dengan yang diinginkan.
Disability is a condition where the body or mind is impaired which makes it more difficult for the person with the condition to perform certain activities and more difficult to interact with the world around them \cite{CDC_2020}. According to the World Health Organization, disability is part of the human person and an integral part of the human experience. It results from the interaction between health conditions such as dementia, blindness or spinal cord injury, and various environmental and personal factors. An estimated 1.3 billion people - or 16\% of the global population - experience significant disability today. This number is growing due to the rise of non-communicable diseases and people living longer \cite{WHO_2023}.

One of the conditions that can cause disability is tetraplegia, which is a spinal cord injury condition where humans can only move the upper body parts, such as the head, neck, and shoulders. One of the diseases included in the tetraplegia condition is “myotrophic lateral sclerosis” or commonly abbreviated as ALS. ALS, previously known as Lou Gehrig's disease, is a neurological disorder that affects motor neurons, which are nerve cells in the brain and spinal cord that control muscle movement and breathing. When motor neurons degenerate and die, they stop sending messages to the muscles, which causes the muscles to weaken, start twitching (fasciculation) and shrink (atrophy). Eventually, in people with ALS, the brain loses its ability to initiate and control the muscle movements needed for walking, speaking, chewing and other functions, as well as breathing. ALS is progressive, which means the symptoms worsen over time \cite{NINDS}. 

In the last stage of ALS disease, which on average takes two to five years \cite{ALS_2023}, ALS patients usually lose the ability to make physical movements including speaking and handwriting, but fortunately they can still move their eyes \cite{Eyesay_2023}. Therefore, ALS patients need a companion to perform their daily activities, but there comes a time when they need to be able to move independently. In that case, they need a wheelchair to move around in doing their daily activities. On the other hand, current electric wheelchairs rely on the user's upper arm for control, which makes it difficult for tetraplegia patients to control them \cite{9935646}. 

To overcome these problems, it is important to find solutions to simplify the wheelchair control system so as to increase the independence of ALS patients. A promising approach is the utilization of computer vision technology and embedded systems. Computer vision allows computers to “see” and process images \cite{TIAN20201}, this technology uses cameras to identify, track, and measure targets for further image processing. Computer vision provides the ability to recognize and understand the surrounding environment. It is important to use appropriate hardware for the purpose. Recently, image recognition models have been embedded on actual target platforms, which are special-purpose computers, such as mobile IoT devices or autonomous vehicles, rather than on servers or PCs used in laboratories. Embedded systems are generally used in the field, but there are operational requirements for the environment, physical size, etc \cite{8939843}. By combining these two technologies, this research provides a solution by developing a wheelchair movement control system that can be controlled through eye movements.

In realizing the solution, this research is focused on developing a wheelchair control system that can interact with computer vision technology in an embedded system. NUC\textit(Next Unit of Computing) is the right choice for embedded system development due to its cost effectiveness, ease of use, and flexibility. This research is expected to create an efficient and responsive wheelchair control system in addressing mobility issues in ALS sufferers.